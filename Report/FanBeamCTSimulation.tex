% !TEX TS-program = pdflatex
%\documentclass[draftcls, onecolumn, journal]{IEEEtran}
\documentclass[journal]{IEEEtran}
%\documentclass[a4paper,11pt]{article}
%\usepackage{fullpage}

%\renewcommand{\baselinestretch}{1.9}
\usepackage[hidelinks]{hyperref}
\usepackage{graphicx}
\usepackage{color}
\usepackage{amsmath}
%\usepackage{cite}
\usepackage[
style=ieee,
sorting=ynt
]{biblatex}
\addbibresource{sources.bib}

\newcommand{\argmax}[1]{\underset{#1}{\operatorname{arg}\,\operatorname{max}}\;}

%\bibliographystyle{IEEEtran}

%%%%%%%%%%%%%%%%%%%%%%%%%%%%%%%%%%%%%%%%%%%%%%%%%%%%%%%%%%%%%%%%%%%%%%
\title{Fan-Beam Computerized Tomography Simulation}

\author{Kutay Ugurlu}

%%%%%%%%%%%%%%%%%%%%%%%%%%%%%%%%%%%%%%%%%%%%%%%%%%%%%%%%%%%%%%%%%%%%%%
\begin{document}
%\renewcommand{\baselinestretch}{1.6}

\maketitle

\begin{abstract}This project report demonstrates the implementation of Fan Beam Computerized Tomography simulation. The effect of different design parameters including the length of the detector, the number of beams and the angle between consecutive projections is inspected and discussed comparatively in both quantitative and qualitative manner. The work is derived from the previously developed code in Parallel Beam X-Ray Computerized Tomography \cite{ugurlu2021}. The developed code and GUI to run it can be found in \href{https://github.com/kutay-ugurlu/Fan-Beam-Computerized-Tomography-Simulation}{github.com/kutay-ugurlu/Fan-Beam-Computerized-Tomography-Simulation} \\
%\textit{Keywords:} Inverse electrocardiography, electrocacardiographic imaging, statistical estimation, Bayesian estimation, Kalman filter.
\end{abstract}
\begin{IEEEkeywords}
	imaging, medical imaging, X-Ray computerized tomography, image reconstruction
\end{IEEEkeywords}

\section{Introduction}

\section{Theory}

\section{Implementation}

\section{Results}

\section{Discussion}

\printbibliography

\end{document}

