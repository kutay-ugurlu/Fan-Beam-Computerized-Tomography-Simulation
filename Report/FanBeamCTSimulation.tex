% !TEX TS-program = pdflatex
%\documentclass[draftcls, onecolumn, journal]{IEEEtran}
\documentclass[journal]{IEEEtran}
%\documentclass[a4paper,11pt]{article}
%\usepackage{fullpage}

%\renewcommand{\baselinestretch}{1.9}
\usepackage[hidelinks]{hyperref}
\usepackage{graphicx}
\usepackage{color}
\usepackage{amsmath}
\usepackage{caption}
%\usepackage{cite}
\usepackage[
style=ieee,
sorting=ynt
]{biblatex}
\graphicspath{images/}
\addbibresource{sources.bib}

\newcommand{\argmax}[1]{\underset{#1}{\operatorname{arg}\,\operatorname{max}}\;}

%\bibliographystyle{IEEEtran}

%%%%%%%%%%%%%%%%%%%%%%%%%%%%%%%%%%%%%%%%%%%%%%%%%%%%%%%%%%%%%%%%%%%%%%
\title{Fan-Beam Computerized Tomography Simulation}

\author{Kutay Ugurlu}

%%%%%%%%%%%%%%%%%%%%%%%%%%%%%%%%%%%%%%%%%%%%%%%%%%%%%%%%%%%%%%%%%%%%%%
\begin{document}
%\renewcommand{\baselinestretch}{1.6}

\maketitle

\begin{abstract}This project report demonstrates the implementation of Fan Beam Computerized Tomography simulation. The effect of different design parameters including the length of the detector, the number of beams and the angle between consecutive projections is inspected and discussed comparatively in both quantitative and qualitative manner. The work is derived from the previously developed code in Parallel Beam X-Ray Computerized Tomography \cite{ugurlu2021}. The developed software and GUI to run it can be found in \href{https://github.com/kutay-ugurlu/Fan-Beam-Computerized-Tomography-Simulation}{github.com/kutay-ugurlu/Fan-Beam-Computerized-Tomography-Simulation} \\
%\textit{Keywords:} Inverse electrocardiography, electrocacardiographic imaging, statistical estimation, Bayesian estimation, Kalman filter.
\end{abstract}
\begin{IEEEkeywords}
	imaging, medical imaging, X-Ray computerized tomography, image reconstruction
\end{IEEEkeywords}

\section{Introduction} \label{sec:intro}
The purpose of this project report is to demonstrate the procedure followed to simulate Fan-Beam Projected X-Ray Computerized Tomography. This project report consists of \nameref{sec:theory}, \nameref{sec:implementation}, \nameref{sec:results} and \nameref{sec:discuss} sections. The second section introduces the technical background for the CT simulation and the following section illustrates the algorithm using pseudocode snippets. \nameref{sec:results} and \nameref{sec:discuss} section presents the comparative results regarding different user-specified parameters with the conclusion and reasons behind them.

\subsection{History}
The history of X-Ray Computerized Tomography can be dated back to 1917, when an Austrian mathematician called Johann Radon invented an algorithm, referred to as Radon transform today, on how to calculate line integrals in a two-dimensional section. The idea of computed tomography was developed in 1967 and was first used in a medical setting was in 1971 \cite{richmond2004sir}, by Godfrey Hounsfield. The device was tested at
James Ambrose’s department at Atkinson Morley Hospital in Wimbledon. This first model did not include a computer, instead the waves was written on a magnetic tape of the device EMI Scanner CT1010 in Figure \ref{fig:CT1010}. It was in 1973 that commercial CT scanners were available to the public. \cite{CTHist}

\begin{figure}[h]
\centering
\includegraphics[width=0.2\textwidth]{images/CT.jpg}
\centering \caption{First EMI Scanner \cite{emict}}\label{fig:CT1010}
\end{figure}

\vfill{\null}

\section{Theory} \label{sec:theory}
\subsection{X-Ray Attenuation}
In X-ray tomography, images are modelled as attenuation coefficient distributions which is a measure of how much X-ray beams are attenuated when they propagate through an object. This problem can be modeled as in Eqn. \ref{eq:radon3d} for an arbitrary object.
\begin{equation} 
	I_{measured} = I_0 e^{-\iiint\limits_{object}\mu(x,y,z)dxdydz}
	\label{eq:radon3d}
\end{equation}
When the object to be imaged is two dimensional or can be reduced to a two dimensional slice, Eqn \ref{eq:radon3d} reduces to Eqn \ref{eq:radon2d}:
\begin{equation}
	I_{measured} = I_0 e^{-\iint\limits_{slice}\mu(x,y)dxdy}
	\label{eq:radon2d}
\end{equation}
\subsection{Radon Transform}
Radon Transform computes the line integrals along the objects to obtain projections along an arbitrary angle $\theta$ for an arbitrary beam t, using the formula given in Eqn. \ref{eq:radongeneral}.
\begin{equation}
	p_{\theta}(t) = \iint\limits_{-\infty}^{\ \ \ \infty}\mu(x,y)\delta(xcos(\theta)+ysin(\theta)-t)dxdy
	\label{eq:radongeneral}
\end{equation}
This equation models the X-ray beams as parallel lines through the object. In a more practical scenario, the X-ray source is modelled as a point source and beams are projected from source to the object in fan beam shape, due to the equiangular spaced discrete detector locations. This modelling can be achieved by introducing geometric transformation between projection variables. 

\begin{figure}[h]
\centering
\includegraphics[width=0.4\textwidth]{images/PvsB.jpg}
\caption{Parallel Beams and Fan Beams \cite{zeng2017image}}\label{fig:PvsB}
\end{figure}

In Figure \ref{fig:PvsB}b, the projection angle with respect to center of rotation is defined as $\beta$ and the deviation from the center beam that is parallel to the $\beta$ beam is defined as $\gamma$ angles. In addition, the source to origin distance is labelled as $D$, resulting in source to detector distance of $2D$. 

With the quantities defined above one could transform the equation in \ref{eq:radongeneral} to \ref{eq:transformedgeneral} using Eqn. \ref{eq:trans1} and \ref{eq:trans2}.
\begin{align}
	t &= D \cdot sin(\gamma) \label{eq:trans1} \\
	\theta &= \beta + \gamma \label{eq:trans2} \\
	p_{\beta}(\gamma) &= \iint\limits_{-\infty}^{\ \ \ \infty}\mu(x,y)\delta(xcos(\theta)+ysin(\theta) \label{eq:transformedgeneral} \\&-Dsin(\gamma))d\gamma d\beta \nonumber 
\end{align}
\subsection{Back Projection}



\section{Implementation} \label{sec:implementation}

\section{Results} \label{sec:results}

\section{Discussion} \label{sec:discuss}

\printbibliography

\end{document}

